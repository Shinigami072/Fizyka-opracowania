\documentclass{fizraport}

\authorA{Krzysztof Stasiowski}
\authorB{Joanna Binek}
\team{2}{2a}{1}

\topic{Opracowanie danych pomiarowych}{0}
\carryOutDate{03.10.2018 r.}
\ftHandInDate{10.10.2018 r.}

\begin{document}

\maketitle

\section{Cel ćwiczenia}
Zaznajomienie się
 z typowymi metodami opracowania danych pomiarowych
przy wykorzystaniu wyników pomiarów dla wahadła prostego.

\section{Wstęp}
Wahadło proste jest, jak wskazuje jego nazwa, układem mechanicznym charakteryzującym się prostotą tak  eksperymentu  jak  i  opisu  teoretycznego.
Dlatego nadaje  się dobrze  na ćwiczenie  wprowadzające (zerowe),   mające   na   celu   poznanie   podstawowych   metod   opracowania danych   pomiarowych.
Interpretacja  wyników  opiera  się na  równaniu  określającym  okres  drgań $T$ jako  funkcję długości wahadła $l$ oraz przyspieszenia ziemskiego $g$, 
%
\[ T = 2\pi \sqrt{\frac{l}{g}} \]
%
Wzór ten jest słuszny, jeżeli wychylenie ciężarka z położenia równowagi jest małe.  
Przekształcając powyższy wzór możemy wyznaczyć wartość przyspieszenia ziemskiego:
%
\[ g = \frac{\pi^2l}{T^2} \]
\pagebreak
\section{Opis Doświadczenia}

\pagebreak
\section{Opracowanie Wyników}
\newpage
\section{Podsumowanie}
\newpage
\end{document}


























