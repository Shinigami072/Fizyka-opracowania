\documentclass{fizraport}
\authorA{Krzysztof Stasiowski}
\authorB{Joanna Binek}
\usepackage[version=4]{mhchem}
\usepackage{adjustbox}

\team{2}{2a}{1}

\topic{Współczynnik załamania ciał stałych}{51}{Wyznaczenie współczynnika załamania światła dla płytki szklanej i~pleksiglasowej metodą pomiaru grubości pozornej płytki przy~pomocy mikroskopu.}
\carryOutDate{14.11.2018 r.}
\ftHandInDate{21.11.2018 r.}
\begin{document}
\maketitle
\section{Wstęp teoretyczny}
Gdy wiązka światła przechodzi przez dwa ośrodki o~różnych własnościach optycznych, to~na~powierzchni granicznej częściowo zostaje odbita, częściowo zaś przechodzi do~drugiego
środowiska, ulegając załamaniu. Warunkuje to~prawo załamania, które ma postać:
\begin{equation}
\frac{\sin\alpha}{\sin\beta}=n
\end{equation}
gdzie $\alpha$ -~kąt padania, $\beta$ -~kąt załamanej wiązki światła.

Wielkość $n$~jest~stałą, zwaną współczynnikiem załamania ośrodka 2 względem ośrodka 1. Współczynnik załamania zależy od~długości fali światła padającego.
Prawa odbicia i~załamania są słuszne dla całego widma fal elektromagnetycznych. Z zasady Huygensa wynika, że współczynnik załamania $n$~jest~stosunkiem prędkości światła w~każdym z~ośrodków:
\begin{equation}
n=\frac{V_1}{V_2}
\end{equation}
Z tego względu załamanie może być wykorzystane do~rozłożenia wiązki światła na~składowe o~różnych długościach fali co jest~widoczne na~przykład w~pryzmacie.

\newpage
\addtocounter{page}{2}

\section{Układ pomiarowy}
W skład układu pomiarowego wchodzą: 
\begin{enumerate}
    \item Mikroskop wyposażony w~czujnik mikrometryczny i~nasadkę krzyżową.
    \item Śruba mikrometryczna. 
    \item Zestaw płytek szklanych i~z~pleksiglasu, różnej grubości. 
\end{enumerate}


\addplot[c]{
a) mikroskop i~jego~elementy: 
1 – kondensor,
2 – obiektyw, 
3 – okular,
4 – lusterko lub~lampka oświetleniowa,
5 – czujnik mikrometryczny, którego stopka spoczywa na~ruchomej części mikroskopu,
6 – nasadka krzyżowa $X Y$ mocująca z~pokrętłami do~przesuwu płytki,
7a – pokrętło służące do~przesuwu stolika ruchem zgrubnym,
7b –pokrętło służące do~przesuwu stolika ruchem dokładnym;
b) zasada powstawania obrazu ($A''$) przedmiotu. ($A$).}{mikro}{mikroskop.png}{\textwidth}

\section{Wykonanie ćwiczenia}
Do doświadczenia wykorzystaliśmy dwie płytki z~różnych materiałów:szkła i~plexiglasu. Za pomocą śruby mikrometrycznej zmierzyliśmy grubość obu płytek (po trzy pomiary dla każdej płytki) wykorzystując do~pomiaru miejsca zaznaczone kreskami.

Następnie ustawiliśmy badaną płytkę na~stoliku mikroskopu w~uchwycie.\\ Wykonaliśmy serię odczytów wskazówki czujnika mikrometrycznego $a_g$ po~uprzedniej regulacji pokrętłami odpowiednio 7a i~7b ( 7a -~do~uzyskania ostrego obrazu śladów na~powierzchni płytki, 7b -~do~przesuwania stolika mikroskopu do~położenia, w~którym widoczne są ślady).

\section{Wyniki pomiarów}
Wyniki pomiarów zostały załączone w~tabelkach na~kratkach, które wypełniliśmy podczas wykonywania doświadczenia.
\newpage
\addtocounter{page}{2}

\section{Opracowanie wyników pomiarów}
\subsection{Obliczenie wartości współczynnika załamania $n$~dla badanych płytek}
Wzór na~współczynnik załamania $n$:
\begin{equation}
    n=\frac{d}{h}
\end{equation}
gdzie $d$~-~grubość rzeczywista, $h$~-~grubość pozorna.

\begin{equation*}
\begin{split}
n_{szkło}&=1.7060 \\
n_{plexiglas}&=1.6841 \\
n_{plexiglas+filtr czerwony}&=1.6970 \\
n_{plexiglas+filtr niebieski}&=1.6973 
\end{split}
\end{equation*}


\subsection{Obliczenie niepewności złożonej współczynnika załamania}
Wzór na~niepewność złożoną współczynnika załamania $n$:
\begin{equation}
    \label{eq:nniep}
    u(n)=n\cdot\sqrt{\left(\frac{u(d)}{d} \right )^2 + \left(\frac{u(h)}{h} \right )^2}
\end{equation}

\begin{equation*}
\begin{split}
u_{szkło}(n)&=0.0066 \\
u_{plexiglas}(n)&=0.0064 \\
u_{plexiglas+filtr czerwony}(n)&=0.0085 \\
u_{plexiglas+filtr niebieski}(n)&=0.0087 
\end{split}
\end{equation*}

\subsection{Tabela zbiorcza:}% zawierająca wartości grubości rzeczywistej, pozornej i~ich~niepewności oraz wartości współczynników załamania i~ich~niepewności dla badanych płytek}
\begin{itemize}
   \item  $u(d)$ -~niepewność pomiaru grubości rzeczywistej ($d$) typu B oszacowana na~podstawie użytego przyrządu pomiarowego (śruby mikrometrycznej) 
  \item $u(h)$ -~niepewność pomiaru grubości pozornej ($h$) typu A wyliczona ze~wzoru \ref{eq:hniep}
  \item $u(n)$ -~niepewność złożona współczynnika załamania ($n$) wyliczona ze~wzoru \ref{eq:nniep}  
\end{itemize}
\begin{equation}
    \label{eq:hniep}
    u(h) = \sqrt{\frac{ \displaystyle\sum{(h_i-\bar{h})^2}}
    {(n-1)\cdot(n)}}
\end{equation}
\begin{table}[h]
\centering
\caption{Grubości rzeczywiste, pozorne i~wyliczone współczynniki załamania}
\label{tab:wszystko}
\begin{tabular}{|l|l|l|l|l|l|l|}
\hline
\textbf{Materiał}           & $d$[mm]&$u(d)$[mm]& $h$[mm]& $u(h)[mm]$ & $n$~   & $u(n)$ \\ \hline
szkło                       &   4.28 &     0.01 & 2.5086 &     0.0077 & 1.7060 & 0.0066 \\ \hline
plexiglas                   &   4.34 &     0.01 & 2.5770 &     0.0078 & 1.6841 & 0.0064 \\ \hline
plexiglas + filtr czerwony  &   4.34 &     0.01 & 2.558~ &     0.012~ & 1.6970 & 0.0085 \\ \hline
plexiglas + filtr niebieski &   4.34 &     0.01 & 2.557~ &     0.012~ & 1.6973 & 0.0087 \\ \hline
\end{tabular}
\end{table}
\newpage
\section{Podsumowanie}
%\section{Porównanie obliczonych wartości współczynnika $n$~z~wartościami tablicowymi}
\begin{table}[h]
\centering
\caption{Porównanie obliczonych wartości współczynnika $n$~z~wartościami tablicowymi}
\begin{tabular}{|l|l|l|}
\hline
\textbf{Materiał}              & $n$~zmierzone    & $n$~tablicowe \\ \hline
szkło                          & $(1.7060\pm0.0132)$& 1.53        \\ \hline
plexiglas                      & $(1.6841\pm0.0128)$& 1.49        \\ \hline
plexiglas z~filtrem czerwonym  & $(1.6970\pm0.0170)$& 1.49        \\ \hline
plexiglas z~filtrem niebieskim & $(1.6973\pm0.0174)$& 1.49        \\ \hline
\end{tabular}
\end{table}

\begin{equation}
\label{eq:diff}
\begin{split}
\Large
    \Delta n=(n_{plexiglas+filtr czerwony}-n_{plexiglas+filtr niebieski})=1.6970-1.6973&=\\
    &=-0.0003 \\
    U(\Delta n)= \sqrt{ U(n_{plexiglas+filtr czerwony})^2 + U(n_{plexiglas+filtr niebieski})^2} &=0.024\\
    \Delta n < U(\Delta n)
\end{split}
\end{equation}

Wyliczone wartości współczynnika załamania $n$~dla szkła i~plexiglasu (z~filtrami czerwonym i~niebieskim) nie~zgadzają się~w~granicach niepewności rozszerzonej z~wartościami tablicowymi dla tych materiałów o~stałą wartość wynoszącą około 0.2.

Przypuszczalne powody niezgodności wyników:
\begin{enumerate}
    \item Popełnienie błędu przy~pomiarze grubości rzeczywistej płytek (szacowany na~około 0.5 mm)
    \item Subiektywny odczyt ostrości obrazu z~monitora mikroskopu obarczony błędem własnym
\end{enumerate}

Na podstawie wykonanego obliczenia (\ref{eq:diff}) można wnioskować, że~wartość współczynnika załamania $n$~nie~zależy od~długości fali świetlnej.

\end{document}