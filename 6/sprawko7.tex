\documentclass{fizraport}
\authorA{Krzysztof Stasiowski}
\authorB{Joanna Binek}
\usepackage[version=4]{mhchem}
\usepackage{adjustbox}

\team{2}{2a}{1}

\topic{Współczynnik załamania ciał stałych}{51}{Wyznaczenie współczynnika załamania światła dla płytki szklanej i pleksiglasowej metodą pomiaru grubości pozornej płytki przy pomocy mikroskopu.}
\carryOutDate{14.11.2018 r.}
\ftHandInDate{21.11.2018 r.}
\begin{document}
\maketitle
\section{Wstęp teoretyczny}
Gdy wiązka światła przechodzi przez dwa ośrodki o różnych własnościach optycznych, to na powierzchni granicznej częściowo zostaje odbita, częściowo zaś przechodzi do drugiego
środowiska, ulegając załamaniu. Warunkuje to prawo załamania, które ma postać:
\begin{equation}
\frac{\sin\alpha}{\sin\beta}=n
\end{equation}
gdzie $\alpha$ - kąt padania, $\beta$ - kąt załamanej wiązki światła.\\
Wielkość $n$ jest stałą, zwaną współczynnikiem załamania ośrodka 2 względem ośrodka 1. Współczynnik załamania zależy od długości fali światła padającego.
Prawa odbicia i załamania są słuszne dla całego widma fal elektromagnetycznych. Z zasady Huygensa wynika, że współczynnik załamania $n$ jest stosunkiem prędkości światła w każdym z ośrodków:
\begin{equation}
n=\frac{V_1}{V_2}
\end{equation}
Z tego względu załamanie może być wykorzystane do rozłożenia wiązki światła na składowe o różnych długościach fali co jest widoczne na przykład w pryzmacie.
\newpage
\section{Układ pomiarowy}
W skład układu pomiarowego wchodzą: 
\begin{enumerate}
    \item Mikroskop wyposażony w czujnik mikrometryczny i nasadkę krzyżową.
    \item Śruba mikrometryczna. 
    \item Zestaw płytek szklanych i z pleksiglasu, różnej grubości. 
\end{enumerate}


\addplot[c]{
a) mikroskop i jego elementy: 
1 – kondensor,
2 – obiektyw, 
3 – okular,
4 – lusterko lub lampka oświetleniowa,
5 – czujnik mikrometryczny, którego stopka spoczywa na ruchomej części mikroskopu,
6 – nasadka krzyżowa $X Y$ mocująca z pokrętłami do przesuwu płytki,
7a – pokrętło służące do przesuwu stolika ruchem zgrubnym,
7b –pokrętło służące do przesuwu stolika ruchem dokładnym;
b) zasada powstawania obrazu ($A''$) przedmiotu. ($A$).}{mikro}{mikroskop.png}{\textwidth}

\section{Wykonanie ćwiczenia}
Do doświadczenia wykorzystaliśmy dwie płytki z różnych materiałów:szkła i plexiglasu. Za pomocą śruby mikrometrycznej zmierzyliśmy grubość obu płytek (po trzy pomiary dla każdej płytki) 

\end{document}