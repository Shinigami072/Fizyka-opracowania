\section{Zagadnienia Kontrolne}
\begin{enumerate}
    \addtocounter{enumi}{1}%\item Ruch Falowy.
    \addtocounter{enumi}{1}%\item Równanie falowe, fale harmoniczne.
\item Prędkość rozchodzenia fal sprężystych:\\
Prędkość fazowa / Prędkość rozchodzenia się fali jest prędkością punktu o~ustalonej fazie drgania. W przypadku fal sprężystych zależy ona od właściwości ośrodka, w~którym fale się rozchodzą. Dla ośrodka jednorodnego, będącego siałem stałym wzór na prędkość ma formę:
\[ v = \sqrt{\frac{E}{\rho}}\]
$V$-prędkość, $\rho$- gęstość ośrodka, $E$ moduł Younga ośrodka
    \addtocounter{enumi}{1}%\item Interferencja fal.
    \addtocounter{enumi}{1}%\item Fala stojąca, częstotliwości własne.
\item Fala koherentna - spójna.
O Koherencji fal możemy mówić w przypadku interferencji ich. Fale koherentne to takie które różnią się w fazie o stałą wartość. Co umożliwia utworzenie stałej interferencji.
\item Analiza Fouriera
Polega na zastosowaniu szeregu Fouriera w celu wydobycia częstotliwości składowych sygnału.
    \addtocounter{enumi}{1}%\item Drgania prętów, strun i słupów powietrza.

\end{enumerate}