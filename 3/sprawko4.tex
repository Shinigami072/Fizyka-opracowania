\documentclass{fizraport}
\authorA{Krzysztof Stasiowski}
\authorB{Joanna Binek}
\team{2}{2a}{1}

\topic{Fale podłużne w~ciałach stałych}{29}{
Wyznaczenie modułu Younga dla różnych materiałów na podstawie pomiaru prędkości rozchodzenia się fali dźwiękowej w~pręcie. }
\carryOutDate{24.10.2018 r.}
\ftHandInDate{31.10.2018 r.}
\begin{document}
\maketitle

\section{Wstęp teoretyczny}
    Fala podłużna w~pręcie powstaje na skutek chwilowego wychylenia się
    fragmentu pręta z~położenia równowagi i~następujących po nim drgań.
    Drgania te, dzięki sprężystości ośrodka, mogą być przekazywane dalej i~mogą
    rozchodzić się po całym ośrodku. Szybkość rozchodzenia się fali zależy
    od bezwładności i~sprężystości ośrodka, w~którym się rozchodzi.
    Aby znaleźć związek pomiędzy modułem Younga a~prędkością rozchodzenia
    fali rozważmy mały wycinek jednorodnego pręta o~grubości $\delta x$ jak na~\figref{fig:fala1}.
    
    \addplot[c]
    {Wycinek pręta pomiędzy $x$ i~$x+\Delta{x}$.
    Kierunek osi $X$ zgodny z~kierunkiem propagacji fali}
    {fala1}{inne/rys1.png}{0.5\textwidth}
    
    
    Rozpatrując naprężenie $\sigma$ w~punkcie $x$ oraz $x$+$\Delta{x}$ na rysunku, a~także oznaczając średnie przemieszczenie wycinka pręta jako $\Psi(x,t)$ można zapisać równanie ruchu w~postaci
    %
    \begin{equation}
    \label{eq:napr}
    \rho S \Delta{x}\frac{\partial^2\Psi(x,t)}{\partial t^2} = S(\sigma(x+\Delta{x})-\sigma(x))
    \end{equation}
    %
    Gdzie $\rho$ oznacza średnią gęstość, $S$ pole powierzchni przekroju, a~$\sigma$
    naprężenie materiału.
    \pagebreak
    
    Przekształcając równanie (\ref{eq:napr}) (skrócenie $S$ po obu stronach równania,
    podzielenie obu stron przez $\Delta{x}$ i~obliczenie granicy przy $\Delta{x}\rightarrow0$) Prawy człon stał się granicą ilorazu różnicowego:
    \begin{equation*}
        \rho\frac{\partial^2\Psi(x,t)}{\partial t^2} = \frac{\partial\sigma}{\partial x}
    \end{equation*}
    %
    Po zastosowaniu prawa Hooka i~zapisując odkształcenie względne jako pochodna funkcji $\Psi$ po $x$ otrzymaliśmy równanie:
    \begin{equation*}
        \frac{\partial^2\Psi(x,t)}{\partial t^2} = \frac{E}{\rho}\frac{\partial^2\Psi(x,t)}{\partial x^2}
    \end{equation*}
%
    z którego widać, że prędkość rozchodzenia się fali w~pręcie wynosi:
    \begin{equation*}
        v = \sqrt{\frac{E}{\rho}}
    \end{equation*}
%
    a co za tym idzie, moduł Younga jest równy \[E = \rho v^2\]

    Fala padająca i~fala odbita w~pręcie, interferują ze sobą tworząc falę stojącą.
    Odległość między węzłami fali stojącej stanowi połowę jej długości.Korzystając z~tego fakty równanie na prędkość ostatecznie
    otrzymaliśmy jako:
    \begin{equation*}
        v = 2 lf
    \end{equation*}
%
    Co prowadzi to do wyrażenia na moduł Younga w~postaci:
    \begin{equation}
        E = 4\rho l^2 f^2
        \label{eq:S4YoungMod}
    \end{equation}
%
    Opracowane przez nas zagadnienia kontrolne do tego doświadczenia zostały dołączone na osobnych,zapisanych ręcznie kartkach.
\newpage %tutaj zszyjemy te nasze kartki ręczne OK
\addtocounter{page}{2}%żeby nasze kartki były zawarte w numeracji

\section{Układ pomiarowy}
    \begin{enumerate}
    		\item Komputer stacjonarny Dell z~systemem Windows XP z~podpiętym mikrofonem i~zainstalowanym oprogramowaniem Zelscope.
    		\item Zestaw siedmiu prętów o~różnych kształtach i ~materiałach.
    		\item Młotek.
    		\item Przyrządy miernicze suwmiarka, miarka w~rolce o~podziałce 1mm oraz waga Detecto o~dokładności 1g.
    \end{enumerate}
    	
\section{Wykonanie ćwiczenia}
    \begin{enumerate}
        \item Pomiar prętów z~miedzi,stali i~aluminium (długość) oraz odpowiadających im próbek (długość,średnica) za pomocą suwmiarki o~niepewności pomiarowej $0.05$mm oraz miarki w~rolce o~podziałce $1$mm.
        \item Zważanie próbek odpowiadających badanym prętom za pomocą wagi w~laboratorium o~dokładności $1$g, w~celu wyznaczenia gęstości materiałów z~których zrobione są pręty.
        \item Uderzenie młotkiem w~pręt, a~następnie zarejestrowanie częstotliwości drgań harmonicznych przy pomocy mikrofonu i~programu Zelscope.
        \item Powtórzenie czynności z~punktu 3 dla wszystkich badanych prętów.
    \end{enumerate}

\section{Wyniki pomiarów}

    Mimo tego, że wykonaliśmy ćwiczenie dla~sześciu prętów, to niestety tylko trzy serie pobranych pomiarów nadawały się do przeanalizowania. Wynika to z~braku możliwości odczytania częstotliwości drgań fal~harmonicznych~dla takich materiałów jak np.~aluminium podczas tego doświadczenia.
    Z~tego względu poniższe wyniki pomiarowe oraz późniejsze ich~opracowanie zawiera jedynie informacje dotyczące pręta stalowego (o~kształcie prostopadłościennym i~okrągłym) oraz pręta miedzianego i~odpowiadających im~próbek.

    \begin{table}[htb]
        \newlength{\CW}
        \setlength{\CW}{4.0cm}
		\centering
		\caption{Pomiary próbek materiałów}
		\label{tab:mat}
		\begin{tabular}{|c|c|c|c|c|c|}
			\hline
		    \multirow[c]{3}{0.6\CW}{Materiał}&
		    \multirow[b]{3}{0.8\CW}{{ Wymiary podstawy\\ (wartość średnia)}\\$[mm]$}&
		    \multirow[b]{3}{0.3\CW}{Masa próbki\\ $[g]$ }&
		    \multirow[b]{3}{0.3\CW}{Długość próbki\\$[mm]$ }&
		    \multirow[b]{3}{0.3\CW}{Objętość\\ $[cm^3]$}&
		    \multirow[b]{3}{0.3\CW}{Gęstość\\
		    $\left[ \frac{kg}{m^3} \right]$}\\
		    &&&&&\\&&&&&\\&&&&&\\
			\hline
			
			Miedź (okrągła)&
			$d=5.00$& 
			66& 
			386&
			\num{7.58}&%7.579092276785376e-06 m^3
			\num{10773.57}\\%10773.565378528301
			\hline
			
			\multirow{2}{*}{Stal (prostopadł.) }&
			$a=4.77$&
			\multirow{2}{*}{36}&
			\multirow{2}{*}{312}&
			\multirow{2}{*}{\num{7.10}}&%7.0989048e-06 m^3
			\multirow{2}{*}{~~\num{5071.20}}\\%5071.204786405925
			&$b=4.77$&&&&\\
			\hline
				
		\end{tabular}
	\end{table}
	
	Następnie korzystając z~równania~(\ref{eq:S4YoungMod}) obliczyliśmy wartości modułu Younga dla~testowanych\\ materiałów. 

	\vfill
	\pagebreak
	\subsection{Tablice Wyników}
    \begin{table}[htb]
			\centering
			\caption{Nr pręta 1 (miedź)}
			\label{tab:prt1}
			\begin{tabular}{|c|c|c|c|}
				\cline{1-1}
				$l = 1802 mm$ & \multicolumn{3}{c}{}\\
				\hline
				\begin{tabular}{c} Nr harmonicznej  \end{tabular} & \begin{tabular}{c} Częstotliwość $f$ \\ \mbox{[Hz]}  \end{tabular}  & 
				\begin{tabular}{c}	Długość fali $ \lambda$ \\ \mbox{[mm]}  \end{tabular} &
				\begin{tabular}{c} Prędkość fali $\upsilon$ \\ \mbox{[m/s]}  \end{tabular}  \\ 
				\hline
				1 & 1030.30 & 3606.00 & 3713.20 \\%3713.2012
				\hline
				2 & 2060.61 & 1802.00 & 3713.22 \\%3713.21922
				\hline
				3 & 3090.91 & 1201.33 & 3713.21 \\%3713.21321333333
				\hline
				4 & 4121.21 & ~901.00 & 3713.21 \\%3713.21021
				\hline
				5 & 5151.52 & ~720.80 & 3713.22 \\%3713.215616
				\hline
				6 & 6181.82 & ~600.67 & 3713.21 \\%3713.21321333333
				\hline
				7 & 7212.12 & ~514.86 & 3713.21 \\%3713.21149714286
				\hline
			    \multicolumn{3}{|r|}{Prędkość średnia { $\bar{\upsilon}$}: }& 3713.21 \\%3713.212
				\hline
				\multicolumn{3}{|r|}{Moduł Younga { $E$ $[GPa]$}: }& ~~\num{148.55} \\%148.545311132
				\hline

			\end{tabular}
	\end{table}
	\vfill
	\begin{table}[htb]
			\centering
			\caption{Nr pręta 2 (stal okrągła)}
			\label{tab:prt2}
			\begin{tabular}{|c|c|c|c|}
				\cline{1-1}
				$l = 1801 mm$ & \multicolumn{3}{c}{}\\
				\hline
				\begin{tabular}{c} Nr harmonicznej  \end{tabular} & \begin{tabular}{c} Częstotliwość $f$ \\ \mbox{[Hz]}  \end{tabular}  & 
				\begin{tabular}{c}	Długość fali $ \lambda$ \\ \mbox{[mm]}  \end{tabular} &
				\begin{tabular}{c} Prędkość fali $\upsilon$ \\ \mbox{[m/s]}  \end{tabular}  \\ 
				\hline
				1 & 1394.94 & 3602.00 & 5024.57 \\%5024.57388
				\hline
				2 & 2878.79 & 1801.00 & 5184.70 \\%5184.70079
				\hline
				3 & 4303.03 & 1200.67 & 5166.50 \\%5166.50468666667
				\hline
				4 & 5696.97 & ~900.50 & 5130.12 \\%5130.121485
				\hline
				5 & 7212.12 & ~720.40 & 5195.61 \\%5195.611248
				\hline
				\multicolumn{3}{|r|}{Prędkość średnia { $\bar{\upsilon}$}: }& 5140.30 \\%5140.302
				\hline
				\multicolumn{3}{|r|}{Moduł Younga { $E$ $[GPa]$}: }& ~~\num{134.00} \\%133.995
				\hline
			\end{tabular}
	
	\end{table}
	\vfill
	\begin{table}[htb]
		\centering
		\caption{Nr pręta 3 (stal prostopadłościenna)}
		\label{tab:prt3}
		\begin{tabular}{|c|c|c|c|}
			\cline{1-1}
			$l = 1803 mm$ & \multicolumn{3}{c}{}\\
			\hline
			\begin{tabular}{c} Nr harmonicznej  \end{tabular} & \begin{tabular}{c} Częstotliwość $f$ \\ \mbox{[Hz]}  \end{tabular}  & 
			\begin{tabular}{c}	Długość fali $ \lambda$ \\ \mbox{[mm]}  \end{tabular} &
			\begin{tabular}{c} Prędkość fali $\upsilon$ \\ \mbox{[m/s]}  \end{tabular}  \\ 
			\hline
			1 & 1393.94 & 3606.00 & 5026.55 \\%5026.54764
			\hline
			2 & 3878.79 & 1803.00 & 6993.46 \\%6993.45837
			\hline
			3 & 4303.03 & 1202.00 & 5172.24 \\%5172.24206
			\hline
			4 & 5696.97 & ~901.50 & 5135.82 \\%5135.818455
			\hline
			\multicolumn{3}{|r|}{Prędkość średnia { $\bar{\upsilon}$}: }& 5582.02 \\%5582.017
			\hline
			\multicolumn{3}{|r|}{Moduł Younga { $E$ $[GPa]$}: }& ~~\num{158.01} \\%158.013211866
			\hline
		\end{tabular}
	\end{table}
	\vfill
    \pagebreak
    \newpage %tutaj zszyjemy te nasze kartki ręczne z pomiarami? tak myślę że to dobre miejsce na kartki z naszymi pomiarami ;)
    \addtocounter{page}{2}%żeby nasze kartki były zawarte w
\subsection{Niepewności pomiarowe}
Do obliczeń przyjęliśmy następujące niepewności:
\begin{enumerate}
    \item niepewność długości pręta $u(l)=1 mm$
    \item niepewność długości promienia $u(r)=0.05 mm$
    \item niepewność masy próbek $u(m)=1 g$
    \item niepewność częstotliwości drgań $u(f)=120 Hz$
    \item niepewność długości fali $u(\lambda)=2 mm$

\end{enumerate}

Zastosowane wzory:
\begin{enumerate}
    \item Wzór na niepewność gęstości:\\
    \begin{align*}
    \small
    u(\rho)&=\sqrt{\left(\frac{\partial \rho}{\partial m}u(m)\right)^2+\left(\frac{\partial \rho}{\partial l}u(l)\right)^2+\left(\frac{\partial \rho}{\partial r}u(r)\right)^2}=\\ &= \sqrt{\left(\frac{1}{l\Pi r^2}u(m)\right)^2+\left(\frac{-m}{l^2 \Pi r^2}u(l)\right)^2+\left(\frac{-2m}{l\Pi r^3}u(r)\right)^2}
    \end{align*}\\
    \begin{align*}
    \small
    u(\rho)&=\sqrt{\left(\frac{
    \partial \rho}
    {\partial m}u(m)\right)^2+
    \left(\frac{\partial \rho}
    {\partial l}u(l)\right)^2+
    \left(\frac{\partial \rho
    }{\partial a}u(a)\right)^2+
    \left(\frac{\partial \rho
    }{\partial b}u(b)\right)^2}=\\&=
    %
    \sqrt{\left(\frac{
    1}
    {lab}u(m)\right)^2+
    \left(\frac{-m}
    {l^2ab}u(l)\right)^2+
    \left(\frac{-m
    }{la^2b}u(a)\right)^2+
    \left(\frac{-m
    }{lab^2}u(b)\right)^2}
    \end{align*}
    \item Wzór na niepewność prędkości fali:\\
    \[\small u(v)=\sqrt{\left(\frac{\partial v}{\partial f}u(f)\right)^2+\left(\frac{\partial v}{\partial \lambda}u(\lambda)\right)^2}=\sqrt{\left(\lambda u(f)\right)^2+\left(f u(\lambda)\right)^2}\]
	
	\item Wzór na niepewność Modułu Younga:\\
	\[\small u(E)=\sqrt{\left(\frac{\partial E}{\partial \rho}u(\rho)\right)^2+\left(\frac{\partial E}{\partial v}u(v)\right)^2} =
	\sqrt{\left(v^2 u(\rho)\right)^2+\left(2 \rho v u(v)\right)^2}\]
\end{enumerate}

\subsubsection{Wyniki dla~pręta miedzianego}
\begin{align*}
\small
 u(\rho)&=33.75 kg/m^3\\%33.74625
 u(v)&= 160.45 m/s\\
u(E)&= 19.33 GPa%7.227812
\end{align*}

\subsubsection{Wyniki dla pręta stalowego}
 \[\small u(\rho)= 459.67 kg/m^3\]%459.6731
 a) okrągłego\\
 \begin{align*}
 \small
 u(v)&=  472.87 m/s\\
u(E)&= 27.49 GPa\\%6.311946 
\end{align*}
b) prostopadłościennego\\
\begin{align*}
\small
 u(v)&= 225.51 m/s\\
u(E)&= 19.19 GPa\\%7.127572WTF tu ciągle mi wychodzi co innego-  w zależności od tego czy python czy R
\end{align*}

\subsection{Porównanie wyników}
\begin{table}[htb]
\caption{Porównanie wyników z wartościami tablicowymi}
\begin{tabular}{|l|l|l|l|}
\hline
Materiał         & Wyliczony moduł Younga & Niepewność rozszerzona k=2 & \begin{tabular}[c]{@{}l@{}}Wartość tablicowa\\ modułu Younga\end{tabular} \\ \hline
miedź            & 148.55 & 38.66 & 110-135
\\\hline
stal okrągła     & 134.00 & 54.98 & 205-210
\\\hline
stal prostopadł. & 158.01 & 38.38 & 205-210
\\\hline
\end{tabular}
\end{table}

	
\section{Wnioski}
    Wyliczone przez nas~wartości modułu Younga dla~miedzi są w~granicach niepewności rozszerzonej zgodne z~wartościami~tablicowymi.

    Natomiast wyliczone przez~nas wartości modułu Younga dla drutów stalowych nie są w~granicach niepewności rozszerzonej zgodne z~wartościami~tablicowymi.
    
    Biorąc pod~uwagę wartości uzyskane podczas pomiarów  w~\tabref{tab:prt2} i~\tabref{tab:prt3}, przyczynami rozbieżności między obliczonymi przez nas~wynikami, a~wartościami oczekiwanymi mogą być:
    \begin{itemize}
        \item błąd w~odczytywaniu wartości częstotliwości dla ~kolejnych Harmonicznych
        \item brak "`wyczucia"' przy~uderzaniu pręta 
        \item brak interferencji z~innymi falami
        \item ogólny stan prętów, zakłócenia z~otoczenia
    \end{itemize}
    
\end{document}