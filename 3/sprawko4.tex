\documentclass{fizraport}
\authorA{Krzysztof Stasiowski}
\authorB{Joanna Binek}
\team{2}{2a}{1}

\topic{Fale podłużne w~ciałach stałych}{29}{
Wyznaczenie modułu Younga dla różnych materiałów na podstawie pomiaru prędkości rozchodzenia się fali dźwiękowej w~pręcie. }
\carryOutDate{24.10.2018 r.}
\ftHandInDate{31.10.2018 r.}
\begin{document}
\maketitle

\section{Wstęp teoretyczny}
Fala podłużna w~pręcie powstaje na skutek chwilowego wychylenia się
fragmentu pręta z~położenia równowagi i~następujących po nim drgań.
Drgania te, dzięki sprężystości ośrodka, mogą być przekazywane dalej i~mogą
rozchodzić się po całym ośrodku. Szybkość rozchodzenia się fali zależy
od bezwładności i~sprężystości ośrodka, w~którym się rozchodzi.
Aby znaleźć związek pomiędzy modułem Younga a~prędkością rozchodzenia
fali rozważmy mały wycinek jednorodnego pręta o~grubości $\delta x$ jak na~\figref{fig:fala1}.

\addplot[c]
{Wycinek pręta pomiędzy $x$ i~$x+\Delta{x}$.
Kierunek osi $X$ zgodny z~kierunkiem propagacji fali}
{fala1}{inne/rys1.png}{0.5\textwidth}


Rozpatrując naprężenie $\sigma$ w~punkcie $x$ oraz $x$+$\Delta{x}$ na rysunku, a~także oznaczając średnie przemieszczenie wycinka pręta jako $\Psi(x,t)$ można zapisać równanie ruchu w~postaci
%
\begin{equation}
\label{eq:napr}
\rho S \Delta{x}\frac{\partial^2\Psi(x,t)}{\partial t^2} = S(\sigma(x+\Delta{x})-\sigma(x))
\end{equation}
%
Gdzie $\rho$ oznacza średnią gęstość, $S$ pole powierzchni przekroju, a~$\sigma$
naprężenie materiału.
\pagebreak

Przekształcając równanie (\ref{eq:napr}) (skrócenie $S$ po obu stronach równania,
podzielenie obu stron przez $\Delta{x}$ i~obliczenie granicy przy $\Delta{x}\rightarrow0$) Prawy człon stał się granicą ilorazu różnicowego:
\begin{equation*}
    \rho\frac{\partial^2\Psi(x,t)}{\partial t^2} = \frac{\partial\sigma}{\partial x}
\end{equation*}
%
Po zastosowaniu prawa Hooka i~zapisując odkształcenie względne jako pochodna funkcji $\Psi$ po $x$ otrzymaliśmy równanie:
\begin{equation*}
    \frac{\partial^2\Psi(x,t)}{\partial t^2} = \frac{E}{\rho}\frac{\partial^2\Psi(x,t)}{\partial x^2}
\end{equation*}
%
z którego widać, że prędkość rozchodzenia się fali w~pręcie wynosi:
\begin{equation*}
    v = \sqrt{\frac{E}{\rho}}
\end{equation*}
%
a co za tym idzie, moduł Younga jest równy \[E = \rho v^2\]

Fala padająca i~fala odbita w~pręcie, interferują ze sobą tworząc falę stojącą.
Odległość między węzłami fali stojącej stanowi połowę jej długości.Korzystając z~tego fakty równanie na prędkość ostatecznie
otrzymaliśmy jako:
\begin{equation*}
    v = 2 lf
\end{equation*}
%
Co prowadzi to do wyrażenia na moduł Younga w~postaci:
\begin{equation}
    E = 4\rho l^2 f^2
    \label{eq:S4YoungMod}
\end{equation}
%
Opracowane przez nas zagadnienia kontrolne to tego doświadczenia zostały dołączone na osobnych,zapisanych ręcznie kartkach.
\newpage %tutaj zszyjemy te nasze kartki ręczne OK
\addtocounter{page}{2}%żeby nasze kartki były zawarte w numeracji

\section{Układ pomiarowy}
\begin{enumerate}
		\item Komputer stacjonarny Dell z~systemem Windows XP z~podpiętym mikrofonem i~zainstalowanym oprogramowaniem Zelscope.
		\item Zestaw siedmiu prętów o~różnych kształtach i materiałach.
		\item Młotek.
		\item Przyrządy miernicze suwmiarka, miarka w~rolce o~podziałce 1mm oraz waga Detecto o~dokładności 1g.
\end{enumerate}
	
\section{Wykonanie ćwiczenia}
\begin{enumerate}
    \item Pomiar prętów z~miedzi,stali i~aluminium (długość) oraz odpowiadających im próbek (długość,średnica) za pomocą suwmiarki o~niepewności pomiarowej $0.05$mm oraz miarki w~rolce o~podziałce $1$mm.
    \item Zważanie próbek odpowiadających badanym prętom za pomocą wagi w~laboratorium o~dokładności $1$g, w~celu wyznaczenia gęstości materiałów z~których zrobione są pręty.
    \item Uderzenie młotkiem w~pręt, a~następnie zarejestrowanie częstotliwości drgań harmonicznych przy pomocy mikrofonu i~programu Zelscope.
    \item Powtórzenie czynności z~punktu 3 dla wszystkich badanych prętów.
\end{enumerate}

\section{Wyniki pomiarów}

Mimo tego, że wykonaliśmy ćwiczenie dla sześciu prętów, to niestety tylko trzy serie pobranych pomiarów nadawały się do przeanalizowania. Wynika to z~braku możliwości odczytania częstotliwości drgania fal harmonicznych dla takich materiałów jak np. aluminium podczas tego doświadczenia.
Z~tego względu poniższe wyniki pomiarowe oraz późniejsze ich opracowanie zawiera jedynie informacje dotyczące pręta stalowego (o~kształcie prostopadłościennym i~okrągłym) oraz pręta miedzianego i~odpowiadających im próbek.

% 	 \multirow{3}{*}{Materiał} & \begin{tabular}{c}  \\ \mbox{[mm]}  \end{tabular}  &  \begin{tabular}{c} \\ \mbox{[g]}\end{tabular} & \begin{tabular}{c} Długość próbki  \\ \mbox{[mm]}  \end{tabular}  & 
% 			\begin{tabular}{c}	Objętość\\ \mbox{[$cm^3$]}  \end{tabular} & 
% 			\begin{tabular}{c}	Gęstość \\ $ \left [\frac{kg}{m^3} \right ] $ \end{tabular}  \\

    \begin{table}[htb]
        \newlength{\CW}
        \setlength{\CW}{4.0cm}
		\centering
		\caption{Pomiary próbek materiałów}
		\label{tab:mat}
		\begin{tabular}{|c|c|c|c|c|c|}
			\hline
		    \multirow[c]{3}{0.6\CW}{Materiał}&
		    \multirow[b]{3}{0.8\CW}{{ Wymiary podstawy\\ (wartość średnia)}\\$[mm]$}&
		    \multirow[b]{3}{0.3\CW}{Masa próbki\\ $[g]$ }&
		    \multirow[b]{3}{0.3\CW}{Długość próbki\\$[mm]$ }&
		    \multirow[b]{3}{0.3\CW}{Objętość\\ $[mm]$}&
		    \multirow[b]{3}{0.3\CW}{Gęstość\\
		    $\left[ \frac{kg}{m^3} \right]$}\\
		    &&&&&\\&&&&&\\&&&&&\\
			\hline
			
			Miedź (okrągła)&
			$d=5.00$& 
			66& 
			386&
			???& 
			???\\
			\hline
			
			\multirow{2}{*}{Stal (prostopad.) }&
			$a=4.77$&
			\multirow{2}{*}{36}&
			\multirow{2}{*}{312}&
			\multirow{2}{*}{???}&
			\multirow{2}{*}{???}\\
			&$b=4.77$&&&&\\
			\hline
				
		\end{tabular}
	\end{table}
	\vfill
	\pagebreak
    \begin{table}[htb]
			\centering
			\caption{Nr pręta 1 (miedź)}
			\label{tab:prt1}
			\begin{tabular}{|c|c|c|c|}
				\cline{1-1}
				$l = 1802 mm$ & \multicolumn{3}{c}{}\\
				\hline
				\begin{tabular}{c} Nr harmonicznej  \end{tabular} & \begin{tabular}{c} Częstotliwość $f$ \\ \mbox{[Hz]}  \end{tabular}  & 
				\begin{tabular}{c}	Długość fali $ \lambda$ \\ \mbox{[m]}  \end{tabular} &
				\begin{tabular}{c} Prędkość fali $\upsilon$ \\ \mbox{[m/s]}  \end{tabular}  \\ 
				\hline
				1 & 1030.30 & 3606.00 & ? \\
				\hline
				2 & 2060.61 & 1802.00 & ? \\
				\hline
				3 & 3090.91 & 1201.33 & ? \\
				\hline
				4 & 4121.21 &  901.00 & ? \\
				\hline
				5 & 5151.52 &  720.80 & ? \\
				\hline
				6 & 6181.82 &  600.67 & ? \\
				\hline
				7 & 7212.12 &  514.86 & ? \\
				\hline
			\end{tabular}
	\end{table}
	\begin{table}[htb]
			\centering
			\caption{Nr pręta 2 (stal okrągła)}
			\label{tab:prt2}
			\begin{tabular}{|c|c|c|c|}
				\cline{1-1}
				$l = 1801 mm$ & \multicolumn{3}{c}{}\\
				\hline
				\begin{tabular}{c} Nr harmonicznej  \end{tabular} & \begin{tabular}{c} Częstotliwość $f$ \\ \mbox{[Hz]}  \end{tabular}  & 
				\begin{tabular}{c}	Długość fali $ \lambda$ \\ \mbox{[m]}  \end{tabular} &
				\begin{tabular}{c} Prędkość fali $\upsilon$ \\ \mbox{[m/s]}  \end{tabular}  \\ 
				\hline
				1 & 1394.94 & 3602.00 & ? \\
				\hline
				2 & 2878.79 & 1801.00 & ? \\
				\hline
				3 & 4303.03 & 1200.67 & ? \\
				\hline
				4 & 5696.97 & 900.5 & ? \\
				\hline
				5 & 7212.12 & 900.5 & ? \\
				\hline
				
			\end{tabular}
	
	\end{table}
	\begin{table}[htb]
		\centering
		\caption{Nr pręta 3 (stal prostpadł.)}
		\label{tab:prt3}
		\begin{tabular}{|c|c|c|c|}
			\cline{1-1}
			$l = 1803 mm$ & \multicolumn{3}{c}{}\\
			\hline
			\begin{tabular}{c} Nr harmonicznej  \end{tabular} & \begin{tabular}{c} Częstotliwość $f$ \\ \mbox{[Hz]}  \end{tabular}  & 
			\begin{tabular}{c}	Długość fali $ \lambda$ \\ \mbox{[m]}  \end{tabular} &
			\begin{tabular}{c} Prędkość fali $\upsilon$ \\ \mbox{[m/s]}  \end{tabular}  \\ 
			\hline
			1 & 1393.94 & 3606.0 & ? \\
			\hline
			2 & 3878.79 & 1803.0 & ? \\
			\hline
			3 & 4303.03 & 1202.00 & ? \\
			\hline
			4 & 5696.97 & 901.50 & ? \\
			\hline
		\end{tabular}
	\end{table}
	\vfill
    \pagebreak

\end{document}