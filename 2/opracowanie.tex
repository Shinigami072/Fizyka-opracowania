\section{Opracowanie wyników}
Wykorzystując dane z~obu tabel sporządziliśmy \figref{fig:w1} i \figref{fig:w2}, gdzie symbolami zaznaczyliśmy te punkty, które w~naszej ocenie odbiegają od prostoliniowego
przebiegu. 
\subsection{Wyliczenie modułu Younga}
Następnie korzystając z~metody najmniejszych kwadratów z wykorzystaniem oprogramowania komputerowego otrzymaliśmy proste o~współczynnikach nachylenia $a_{stal}=0.01783253$,$a_{mosiadz}=0.0197891397$ z~niepewnościami  $u(a_{stal})=0.00053340$,$u(a_{mosiadz})=0.0008412996$.\\
I~wykorzystaliśmy je do~obliczenie modułu~Younga($E$) wykorzystując wzór roboczy(\ref{eq:young})
\begin{equation}
\label{eq:young}
    E = \frac{4l}{\pi d^2 a}
\end{equation}
\begin{equation*}
    E_{stal} = \frac{4l_{stal}}{\pi d_{stal}^2 a_{stal}}=\frac{}{}=value
\end{equation*}
\begin{equation*}
    E_{mosiadz} = \frac{4l_{mosiadz}}{\pi d_{mosiadz}^2 a_{mosiadz}}=\frac{}{}=value
\end{equation*}
\subsection{Wyliczenie niepewności modułu~Younga}
Z kolei wyliczyliśmy niepewność Modułu Younga $u(E)$ korzystając z prawa przenoszenia
niepewności względnych ze wzoru dla trzech zmiennych
\begin{equation*}
\begin{split}
    u(E)&=\sqrt{\left (\frac{\partial E}{\partial a}*u(a)\right )^2+\left (\frac{\partial E}{\partial l}*u(l)\right )^2+\left (\frac{\partial E}{\partial d}*u(d)\right )^2}=\\
&=\sqrt{\left(\frac{-4*l}{a^2*\pi *d^2}*u(a)\right)^2+\left(\frac{4}{a*\pi *d^2}*u(l)\right)^2+\left(\frac{-8l}{a*\pi *d^3}*u(d)\right)^2}=\\
&= \sqrt{E^2\left (\frac{-u(a)}{a} \right )^2 + E^2\left (\frac{u(l)}{l} \right )^2 + E^2\left (\frac{-2*u(d)}{d} \right )^2}=\\
&= E\sqrt{\left (\frac{-u(a)}{a} \right )^2 + \left (\frac{u(l)}{l} \right )^2 + \left (\frac{-2*u(d)}{d} \right )^2} \\
\\
u_c(E) &= ???
\end{split}
\end{equation*}
       
    
Wyznaczyliśmy również niepewność rozszerzoną $U(E)$:
\begin{align*}
        U(E) &= k \cdot u_c(E) \\
        U(E) &= 2 \cdot ?? = ??
\end{align*}
%todać gdzieś niżej XD

\subsection{Porównanie wyliczonej wartości Modułu Younga z wartością tablicową dla (wstawi materiał)}
Na podstawie powyższych wyliczeń można stwierdzić, że uzyskana wartość Modułu Younga dla (materiał)  zgadza się/nie zgadza się w granicach
niepewności rozszerzonej z wartością tablicową, która wynosi ... .
\newlength{\colW}
\setlength{\colW}{2.2cm}
\begin{table}[hb]
\centering
\caption{Wyniki Pomiaru drutu stalowego:}
\label{tab:stal}
\begin{flushleft}
Rodzaj materiału: Stal\\
Długość drutu: $l$=1060 mm,~$u(l)$= 1 mm\\
Średnica drutu $d$: 1.29 mm,~1.28 mm,~1.29 mm\\
Średnica drutu średnia: $\Bar{d}=$ 1.29 mm,~$u(\Bar{d})=$ 0.01 mm\\
Współczynnik nachylenia wykresu: $a_{stal}=0.01783253$,~$u(a_{stal})=0.00053340$\\
\end{flushleft}
\begin{tabular}{|r|r|r|r|r|}
    \hline
    \multirow[b]{3}{\colW}{ Masa odważników \\$[kg]$} & 
    \multirow[b]{3}{0.50\colW}{ Siła \\$[N]$} &
    \multirow[b]{3}{0.7\colW}{Wskazanie czujnika  $\uparrow $\\$[mm]$} &
    \multirow[b]{3}{0.7\colW}{Wskazanie czujnika  $\downarrow $\\$mm]$} &
    \multirow[b]{3}{0.5\colW}{Średnie $\Delta l $\\$[mm]$}\\
    &&&&\\
    &&&&
    \csvreader[separator=semicolon, head to column names,]{tab2/tabelkaDlaStali.csv}{}
    {\\\hline \mass & \F & \up & \down & \mean}
    \\\hline
\end{tabular}
\end{table}

\begin{table}[hb]
\centering
\caption{Wyniki Pomiaru drutu mosiężnego:}
\label{tab:mosiadz}
\begin{flushleft}
Rodzaj materiału: Mosiądz\\
Długość drutu: $l$=1610 mm,~ $u(l)$= 1 mm\\
Średnica drutu $d$: 1.19 mm,~1.19 mm,~1.19 mm\\
Średnica drutu średnia: $\Bar{d}=$ 1.19 mm,~ $u(\Bar{d})=$ 0.01 mm\\
Współczynnik nachylenia wykresu: $a_{mosiadz}=0.0197891397$,~$u(a_{mosiadz})=0.0008412996$\\
\end{flushleft}
\begin{tabular}{|r|r|r|r|r|}
    \hline
    \multirow[b]{3}{\colW}{ Masa odważników \\$[kg]$} & 
    \multirow[b]{3}{0.50\colW}{ Siła \\$[N]$} &
    \multirow[b]{3}{0.7\colW}{Wskazanie czujnika  $\uparrow $\\$[mm]$} &
    \multirow[b]{3}{0.7\colW}{Wskazanie czujnika  $\downarrow $\\$[mm]$} &
    \multirow[b]{3}{0.5\colW}{Średnie $\Delta l $\\$[mm]$}\\
    &&&&\\
    &&&&
    \csvreader[separator=semicolon, head to column names,]{tab2/tabelkaMosiadz.csv}{}
    {\\\hline \mass & \F & \up & \down & \mean}
    \\\hline
\end{tabular}
\end{table}



